\documentclass[12pt,twoside]{article}
\usepackage{jmlda}
\usepackage[utf8]{inputenc}
\usepackage[russian]{babel}
\usepackage[T2A]{fontenc}
\usepackage{lineno}
\usepackage{setspace}
\linenumbers
\doublespacing

\usepackage[left=1.5cm,right=1.5cm,
    top=2cm,bottom=2cm,bindingoffset=0cm]{geometry}

%\NOREVIEWERNOTES
\title
    [Поиск символов в художественных изображениях] % Краткое название; не нужно, если полное название влезает в~колонтитул
    {Поиск символов в художественных изображениях}
\author
    {Лемтюжникова~Д.\,В., Апишев~М.\,А., Козинов~А.\,В.} % основной список авторов, выводимый в оглавление
\thanks
    {  
   Научный руководитель:  Стрижов~В.\,В.
   Задачу поставила:  Лемтюжникова~Д.\,В.}
\email
    {kozinov.av@phystech.edu}

\abstract
% TODO Есть ли контекстное распознавание 
%
    {  Рассматривается проблема распознавания художественного изображения, содержащего символы, в зависимости от контекста. Символ можно распознать многими способами, причём в каждом отдельном случае значение выбирается в зависимости от контекста. Контекст связан не только с окружающими фрагментами на картине, но и с историко-культурными событиями, связанными с изображением.


\bigskip
\textbf{Ключевые слова}: \emph {Анализ изображения, CNN,
boxing изображений}.}
\titleEng
    {JMLDA paper example: file jmlda-example.tex}
\authorEng
    {Author~F.\,S.$^1$, CoAuthor~F.\,S.$^2$, Name~F.\,S.$^2$}

\abstractEng
    {This document is an example of paper prepared with \LaTeXe\
    typesetting system and style file \texttt{jmlda.sty}.

    \bigskip
    \textbf{Keywords}: \emph{keyword, keyword, more keywords}.}
\begin{document}
\maketitle
%\linenumbers
\section{Введение}

В данной работе рассматривается задача понимания художественных изображений алгоритмами машинного обучения.  Основная цель - это распознать на изображении ключевые символы, а с помощью инфрмации о них и информации об изображении сформировать текстовое описание.

Cтоит определиться с тем, что такое \textbf{символ}. Существует много определений понятия символ, одно из них следующее: "Символ имеет очень сложное значение, потому что не подчиняется причине; он всегда предполагает много значений, и эта многозначность не может быть сведена к единой логической системе"(В.И. Иванов).  И принято решение определять символ с помощью экспертов. Таким образом сформирована база размеченных изображений с выделенными фрагментам, которым сопоставлено название символа и его значение при данном контексте.


На предложенной выборке производится обучение свёрточной нейронной сети. А далее процесс анализа изображения происходит следующим образом: выделяются основные объёкты и фон, производится классификация полученных объёктов, далее для каждого из объёкта выбирается описание на основе того, как элементы связаны друг с другом и с фоном.

Это решение может быть использовано для оценки стоимости картины перед аукционом. Но в отличии от подкхода \cite{art_appraisal}, который анализирует картину целиком, представленный подход учитывает наличие специальных смысловых единиц~--- символов.


В данной статье рассматривается упрощённая формулировка задачи, в которой нужно определить, присутствует ли символ на изображении или нет.

\section{Постановка задачи}
\subsection{Входные данные}
На вход подаётся $RGB$ изображение $I$ размера $H\times W$. То есть $I$~--- матрица размера $H \times W \times 3$, причём $I_{i, j, k} \in \overline{0, 255}$. Размер изображения заранее не фиксируется и H, и W могут принимать разные значения.

\subsection{Выходные данные}
Для каждого класса следует выделить все фрагменты изображения, на которых изображены объекты соответствующего класса. А именно получить список множеств $B_1, ..., B_N$, где $B_i$~$=~\{BB_{gt, 1}^{(i)},\  ...,\ BB_{gt, a_i}^{(i)}\}$~--множество найденных \textit{Bounding boxes}.
\subsection{Качество решения}
Пусть же набор множеств  $B_1', ..., B_N'$, где $B_i'$~$=\{BB_{l, 1}^{(i)},\ ...,\ BB_{l, b_i}^{(i)} \}$~--- истинный набор \textit{Bounding boxes}(\cite{metrics}). Для определения расстояния между двумя \textit{Bounding boxes}, используя метрику отношения пересечения к объединению(\cite{iou})

$$
IoU(B_l, B_{gt}) = \frac{Area(B_l \cap B_{gt})}{Area(B_l \cup B_{gt})}
$$

Тогда качество предсказания класса $i$ будем вычислять следующим образом:
$$
IoU_i = \frac{1}{b_i}\sum\limits_{j=1}^{b_i}\min\limits_{k=1...a_i}IoU(B_{l, j}^{(i)}, B_{gt, k}^{(i)}
$$

И для подсчёта общего качества решения достаточно просуммировать по всем классам:

$$
IoU = \frac{1}{N}\sum\limits_{i=1}^{N}IoU_i
$$

\bibliography{Kozinov2019Project45}{}
\bibliographystyle{plain}   
\end{document}

